% *======================================================================*
%  Cactus Thorn template for ThornGuide documentation
%  Author: Ian Kelley
%  Date: Sun Jun 02, 2002
%
%  Thorn documentation in the latex file doc/documentation.tex
%  will be included in ThornGuides built with the Cactus make system.
%  The scripts employed by the make system automatically include
%  pages about variables, parameters and scheduling parsed from the
%  relevant thorn CCL files.
%
%  This template contains guidelines which help to assure that your
%  documentation will be correctly added to ThornGuides. More
%  information is available in the Cactus UsersGuide.
%
%  Guidelines:
%   - Do not change anything before the line
%       % START CACTUS THORNGUIDE",
%     except for filling in the title, author, date, etc. fields.
%        - Each of these fields should only be on ONE line.
%        - Author names should be separated with a \\ or a comma.
%   - You can define your own macros, but they must appear after
%     the START CACTUS THORNGUIDE line, and must not redefine standard
%     latex commands.
%   - To avoid name clashes with other thorns, 'labels', 'citations',
%     'references', and 'image' names should conform to the following
%     convention:
%       ARRANGEMENT_THORN_LABEL
%     For example, an image wave.eps in the arrangement CactusWave and
%     thorn WaveToyC should be renamed to CactusWave_WaveToyC_wave.eps
%   - Graphics should only be included using the graphicx package.
%     More specifically, with the "\includegraphics" command.  Do
%     not specify any graphic file extensions in your .tex file. This
%     will allow us to create a PDF version of the ThornGuide
%     via pdflatex.
%   - References should be included with the latex "\bibitem" command.
%   - Use \begin{abstract}...\end{abstract} instead of \abstract{...}
%   - Do not use \appendix, instead include any appendices you need as
%     standard sections.
%   - For the benefit of our Perl scripts, and for future extensions,
%     please use simple latex.
%
% *======================================================================*
%
% Example of including a graphic image:
%    \begin{figure}[ht]
% 	\begin{center}
%    	   \includegraphics[width=6cm]{MyArrangement_MyThorn_MyFigure}
% 	\end{center}
% 	\caption{Illustration of this and that}
% 	\label{MyArrangement_MyThorn_MyLabel}
%    \end{figure}
%
% Example of using a label:
%   \label{MyArrangement_MyThorn_MyLabel}
%
% Example of a citation:
%    \cite{MyArrangement_MyThorn_Author99}
%
% Example of including a reference
%   \bibitem{MyArrangement_MyThorn_Author99}
%   {J. Author, {\em The Title of the Book, Journal, or periodical}, 1 (1999),
%   1--16. {\tt http://www.nowhere.com/}}
%
% *======================================================================*

\documentclass{article}

% Use the Cactus ThornGuide style file
% (Automatically used from Cactus distribution, if you have a
%  thorn without the Cactus Flesh download this from the Cactus
%  homepage at www.cactuscode.org)
\usepackage{../../../../doc/latex/cactus}

\begin{document}

% The author of the documentation
\author{Gabriele Bozzola \textless gabrielebozzola@arizona.edu\textgreater}

% The title of the document (not necessarily the name of the Thorn)
\title{Christoffel}

% the date your document was last changed:
\date{\today}

\maketitle

% Do not delete next line
% START CACTUS THORNGUIDE

% Add all definitions used in this documentation here
%   \def\mydef etc

% Add an abstract for this thorn's documentation
\begin{abstract}
  \texttt{Christoffel} computes the four-dimensional connection coefficients
  using the derivatives of the metric. We compute the spatial derivatives with
  finite difference methods (2nd, 4th, or 6th order), and we read off the time
  derivatives from the right-hand-sides of the evolution equations.
\end{abstract}

% The following sections are suggestive only.
% Remove them or add your own.

\section{\texttt{Christoffel}}

\texttt{Christoffel} is a simple thorn that implements the equation
\begin{equation}
  \label{eq:christoffel}
  \Gamma^{\mu}_{\;\;\alpha\beta} = \frac{1}{2} g^{\mu\nu} \left(g_{\nu\alpha, \beta} + g_{\nu\beta, \alpha} - g_{\alpha\beta, \nu}\right)\,,
\end{equation}
to compute the four-dimensional Christoffel symbols. \texttt{Christoffel}
computes the spatial derivatives using finite difference, the order of which can
be controlled with the parameter \texttt{derivs_order}. The trickier step is to
compute the time derivatives. \texttt{ADMBase} contains the time derivatives of
the lapse and of the shift, but not the one of the metric. In
\texttt{Christoffel}, we use the right-hand-side of the evolution equations to
access the time derivatives of metric. At the moment, \texttt{Christoffel}
supports only \texttt{LeanBSSNMoL}, but it is easy to change that.

The only other option in \texttt{Christoffel} is \texttt{compute_every} that
sets how often the Christoffel symbols have to be computed.

The test included in \texttt{Christoffel} will not work with the upstream
version of \texttt{LeanBSSNMoL}. A patched version with the two following
changes are required:
\begin{enumerate}
  \item The right-hand-sides have to be computed after the initial data too,
  \item If \texttt{lapse_evolution_method} and \texttt{shift_evolution_method} are
  set to \texttt{static}, \texttt{LeanBSSNMoL} must not touch \texttt{dtlapse}
  and \texttt{dtshift}.
\end{enumerate}

\texttt{Christoffel} optionally saves and compute the derivatives of the
four-dimensional metric. This is enabled by the \texttt{save_dgab} parameter.

\texttt{Christoffel} does not do time interpolation, so \texttt{compute_every}
should be set to a value such that all the refinement levels are at the same
time.



% Do not delete next line
% END CACTUS THORNGUIDE

\end{document}
